
### **1. Budgeting**
- **Problem**: You earn ₹50,000 per month and aim to save 20%.  
  **Solution**:  
  \[
  \text{Savings} = \text{Income} \times \text{Savings Percentage}
  \]  
  \[
  \text{Savings} = ₹50,000 \times 0.20 = ₹10,000
  \]  
  Remaining ₹40,000 can be allocated to expenses and other categories.

---

### **2. Loan Repayment**  
- **Problem**: Calculate the EMI for a ₹10,00,000 loan at 8% annual interest for 5 years.  
  **Formula**:  
  \[
  EMI = \frac{P \times r \times (1 + r)^n}{(1 + r)^n - 1}
  \]  
  Where:  
  \(P = ₹10,00,000\), \(r = 0.08 / 12 = 0.00667\), \(n = 5 \times 12 = 60\) months.  

  **Solution**:  
  \[
  EMI = \frac{10,00,000 \times 0.00667 \times (1 + 0.00667)^{60}}{(1 + 0.00667)^{60} - 1}
  \]  
  Approximate calculation yields **₹20,280 per month**.

---

### **3. Investing in Stocks**  
- **Problem**: ₹1,00,000 grows at 15% annually for 5 years.  
  **Formula**:  
  \[
  A = P \times (1 + r)^t
  \]  
  Where \(P = ₹1,00,000\), \(r = 0.15\), \(t = 5\).  

  **Solution**:  
  \[
  A = ₹1,00,000 \times (1 + 0.15)^5 = ₹1,00,000 \times 2.011357 = ₹2,01,136
  \]  
  Your investment grows to **₹2,01,136**.

---

### **4. Compound Interest**  
- **Problem**: ₹1,00,000 at 10% annual interest, compounded monthly, for 10 years.  
  **Formula**:  
  \[
  A = P \times (1 + r/n)^{nt}
  \]  
  Where \(P = ₹1,00,000\), \(r = 0.10\), \(n = 12\), \(t = 10\).  

  **Solution**:  
  \[
  A = ₹1,00,000 \times (1 + 0.10/12)^{12 \times 10} = ₹1,00,000 \times 2.593742 = ₹2,59,374.2
  \]  
  Final value: **₹2,59,374.2**.

---

### **5. Real Estate Investment**  
- **Problem**: ROI for a property costing ₹50,00,000 with ₹5,00,000 annual rental income.  
  **Formula**:  
  \[
  ROI = \frac{\text{Net Profit}}{\text{Cost of Investment}} \times 100
  \]  
  \[
  ROI = \frac{₹5,00,000}{₹50,00,000} \times 100 = 10\%
  \]  
  **ROI is 10%.**

---

### **6. Retirement Planning**  
- **Problem**: To save ₹1 crore at age 60 starting at 30, earning 12% annually.  
  **Formula**:  
  \[
  PMT = \frac{FV \times r}{(1 + r)^n - 1}
  \]  
  Where \(FV = ₹1,00,00,000\), \(r = 0.12 / 12 = 0.01\), \(n = 30 \times 12 = 360\).  

  **Solution**:  
  \[
  PMT = \frac{1,00,00,000 \times 0.01}{(1 + 0.01)^{360} - 1} = ₹1,751.41
  \]  
  Save **₹1,751.41 per month**.

---

### **7. Tax Planning**  
- **Problem**: Taxable income of ₹12,00,000 after deductions.  
  **Solution**: Assuming slabs:  
  - ₹0–₹2,50,000: No tax  
  - ₹2,50,001–₹5,00,000: 5% = ₹12,500  
  - ₹5,00,001–₹10,00,000: 20% = ₹1,00,000  
  - ₹10,00,001+: 30% = ₹60,000  

  Total tax = ₹12,500 + ₹1,00,000 + ₹60,000 = **₹1,72,500**.  
  Effective tax rate = \(₹1,72,500 / ₹12,00,000 = 14.37\%\).

---

### **8. Business Profitability**  
- **Problem**: Break-even for fixed costs ₹5,00,000, variable cost ₹50/unit, selling price ₹100/unit.  
  **Formula**:  
  \[
  \text{Break-even Units} = \frac{\text{Fixed Costs}}{\text{Selling Price} - \text{Variable Cost}}
  \]  
  \[
  \text{Break-even Units} = \frac{₹5,00,000}{₹100 - ₹50} = 10,000 \, \text{units}
  \]  
  Sell **10,000 units** to break even.

---

### **9. Currency Exchange**  
- **Problem**: Exchange ₹82/USD to invest $1,000.  
  **Solution**:  
  \[
  \text{Value in ₹} = \text{USD} \times \text{Exchange Rate} = 1,000 \times 82 = ₹82,000
  \]  
  If the rate changes to ₹85/USD:  
  \[
  \text{Value in ₹} = 1,000 \times 85 = ₹85,000
  \]  
  Gain: **₹3,000**.

---

### **10. Inflation Impact**  
- **Problem**: Inflation is 6% annually, how much is ₹1,00,000 worth in 10 years?  
  **Formula**:  
  \[
  \text{Value} = P \times (1 - r)^t
  \]  
  Where \(P = ₹1,00,000\), \(r = 0.06\), \(t = 10\).  

  **Solution**:  
  \[
  \text{Value} = ₹1,00,000 \times (1 - 0.06)^{10} = ₹1,00,000 \times 0.558 = ₹55,800
  \]  
  Worth: **₹55,800** after 10 years.

---

Let me know if you'd like further assistance with these calculations!
